\documentclass{article}



\usepackage{arxiv}

\usepackage[utf8]{inputenc} % allow utf-8 input
\usepackage[T1]{fontenc}    % use 8-bit T1 fonts
\usepackage{hyperref}       % hyperlinks
\usepackage{url}            % simple URL typesetting
\usepackage{booktabs}       % professional-quality tables
\usepackage{amsfonts}       % blackboard math symbols
\usepackage{nicefrac}       % compact symbols for 1/2, etc.
\usepackage{microtype}      % microtypography
\usepackage{lipsum}		% Can be removed after putting your text content



\title{A MAST analysis of a Ravenscar application example}

%\date{September 9, 1985}	% Here you can change the date presented in the paper title
%\date{} 					% Or removing it

\author{
  Giovanni Jiayi Hu\\
  Department of Mathematics\\
  University of Padua, Italy I-35121\\
  \texttt{Email: giovannijiayi.hu@studenti.unipd.it} \\
  %% examples of more authors
   \And
   Alessio Gobbo \\
   Department of Mathematics\\
   University of Padua, Italy I-35121\\
   \texttt{Email: alessio.gobbo@studenti.unipd.it} \\
  %% \AND
  %% Coauthor \\
  %% Affiliation \\
  %% Address \\
  %% \texttt{email} \\
  %% \And
  %% Coauthor \\
  %% Affiliation \\
  %% Address \\
  %% \texttt{email} \\
  %% \And
  %% Coauthor \\
  %% Affiliation \\
  %% Address \\
  %% \texttt{email} \\
}

% Uncomment to remove the date
%\date{}

% Uncomment to override  the `A preprint' in the header
\renewcommand{\headeright}{Technical Report}
\renewcommand{\undertitle}{Technical Report}

\begin{document}
\maketitle

\begin{abstract}
Buchi del culo. \lipsum[1]
\end{abstract}


% keywords can be removed
% \keywords{First keyword \and Second keyword \and More}

\section{Introduction}

There is increasing recognition that the software components of critical real-time applications must be provably predictable. This is particularly so for a hard real-time system, in which the failure of a component of the system to meet its timing deadline can result in an unacceptable failure of the whole system. The choice of a suitable design and development method, in conjunction with supporting tools that enable the real-time performance of a system to be analysed and simulated, can lead to a high level of confidence that the final system meets its real-time constraints.

The use of Ada has proven to be of great value within high integrity and real-time applications, albeit via language subsets of deterministic constructs, to ensure full analysability of the code. The research work in schedulability analysis has been mapped onto a number of new Ada constructs and rules that have been incorporated into the Real-Time Annex of the Ada language standard [RM D]. This has opened the way for these tasking constructs to be used in high integrity subsets whilst retaining the core elements of predictability and reliability.

The Ravenscar Profile is a subset of the tasking model, restricted to meet the real-time community requirements for determinism, schedulability analysis and memory-boundedness, as well as being suitable for mapping to a small and efficient run-time system that supports task synchronization and communication, and which could be certifiable to the highest integrity levels.

The example presented in this paper is extracted from "Guide for the use of the
Ada Ravenscar Profile in
high integrity systems" \cite{ycs} and it is designed to illustrate the expressive power of the Ravenscar Profile and the associated coding paradigms, which aim to facilitate off-line scheduling analysis.

The extended application example uses all of the concurrency components permitted by the Ravenscar Profile. The structure of the example models, on a reduced and simplified scale, the operation of real-world embedded real-time systems.

The example system includes a periodic process that handles orders for a variable amount of workload. Whenever the request level exceeds a certain threshold, the periodic process farms the excess load out to a supporting sporadic process. While such orders are executed, the system may receive interrupt requests from an external source. Each interrupt treatment records an entry in an activation log. When specific conditions hold, the periodic process releases a further sporadic process to perform a check on the interrupt activation entries recorded in the intervening period. The policy of work delegation adopted by the system allows the periodic process to ensure the constant discharge of a guaranteed level of workload. The correct implementation of this policy also requires assigning the periodic process a higher priority than those assigned to the sporadic processes, so that guaranteed work can be performed in preference to subsidiary activities.

MAST, a Modeling and Analysis Suite for Real-Time Applications, is a model for representing the temporal and logical elements of real-time applications \cite{mast}. This model allows a very rich description of the system, including the effects of event or message-based synchronization, multiprocessor and distributed architectures as well as shared resource synchronization.

\section{FPS analysis}

\begin{table}[!htbp]
  \centering
  \begin{tabular}{llll}
    \toprule
    Transaction & Worst case response time (s) & Slack & Worst blocking time (s)  \\
    \midrule
    rp\_transaction & 0.020393  & 2477.0\% &  2.000E-06  \\
    ocp\_transaction & 0.026525 & 10852.0\% & 1.000E-06 \\
    alr\_transaction & 0.030109 & 27088.7\% & 0.00 \\
    event\_queue\_interrupt & 1.100E-05 & N/A & 1.000E-06 \\
    event\_queue\_interrupt & 3.818E-05 & N/A & 1.000E-06 \\
    \bottomrule
  \end{tabular}
  \caption{Holistic analysis results for FPS}
  \label{tab:holistic-fps}
\end{table}

The system slack is 2401.2\%.

\section{Introduction}
\lipsum[2]
\lipsum[3]


\section{Headings: first level}
\label{sec:headings}

\lipsum[4] See Section \ref{sec:headings}.

\subsection{Headings: second level}
\lipsum[5]
\begin{equation}
\xi _{ij}(t)=P(x_{t}=i,x_{t+1}=j|y,v,w;\theta)= {\frac {\alpha _{i}(t)a^{w_t}_{ij}\beta _{j}(t+1)b^{v_{t+1}}_{j}(y_{t+1})}{\sum _{i=1}^{N} \sum _{j=1}^{N} \alpha _{i}(t)a^{w_t}_{ij}\beta _{j}(t+1)b^{v_{t+1}}_{j}(y_{t+1})}}
\end{equation}

\subsubsection{Headings: third level}
\lipsum[6]

\paragraph{Paragraph}
\lipsum[7]

\section{Examples of citations, figures, tables, references}
\label{sec:others}
\lipsum[8] \cite{kour2014real,kour2014fast} and see \cite{hadash2018estimate}.

The documentation for \verb+natbib+ may be found at
\begin{center}
  \url{http://mirrors.ctan.org/macros/latex/contrib/natbib/natnotes.pdf}
\end{center}
Of note is the command \verb+\citet+, which produces citations
appropriate for use in inline text.  For example,
\begin{verbatim}
   \citet{hasselmo} investigated\dots
\end{verbatim}
produces
\begin{quote}
  Hasselmo, et al.\ (1995) investigated\dots
\end{quote}

\begin{center}
  \url{https://www.ctan.org/pkg/booktabs}
\end{center}


\subsection{Figures}
\lipsum[10]
See Figure \ref{fig:fig1}. Here is how you add footnotes. \footnote{Sample of the first footnote.}
\lipsum[11]

\begin{figure}
  \centering
  \fbox{\rule[-.5cm]{4cm}{4cm} \rule[-.5cm]{4cm}{0cm}}
  \caption{Sample figure caption.}
  \label{fig:fig1}
\end{figure}

\subsection{Tables}
\lipsum[12]
See awesome Table~\ref{tab:table}.

\begin{table}
 \caption{Sample table title}
  \centering
  \begin{tabular}{lll}
    \toprule
    \multicolumn{2}{c}{Part}                   \\
    \cmidrule(r){1-2}
    Name     & Description     & Size ($\mu$m) \\
    \midrule
    Dendrite & Input terminal  & $\sim$100     \\
    Axon     & Output terminal & $\sim$10      \\
    Soma     & Cell body       & up to $10^6$  \\
    \bottomrule
  \end{tabular}
  \label{tab:table}
\end{table}

\subsection{Lists}
\begin{itemize}
\item Lorem ipsum dolor sit amet
\item consectetur adipiscing elit.
\item Aliquam dignissim blandit est, in dictum tortor gravida eget. In ac rutrum magna.
\end{itemize}


\bibliographystyle{unsrt}
%\bibliography{references}  %%% Remove comment to use the external .bib file (using bibtex).
%%% and comment out the ``thebibliography'' section.


%%% Comment out this section when you \bibliography{references} is enabled.
\begin{thebibliography}{1}

\bibitem{ycs}
A Burns, B Dobbing, T Vardanega.
\newblock Guide for the use of the Ada Ravenscar Profile in high integrity systems.
\newblock In {\em University of York Technical Report YCS-2003-348}. January 2003.

\bibitem{mast}
M. GonzAlez Harbour, J.J. GutiCrrez Garcia, J.C. Palencia GutiCrrez, and J.M. Drake Moyano.
\newblock MAST Modeling and Analysis Suite for Real Time Applications.
\newblock In {\em Proceedings 13th Euromicro Conference on Real-Time Systems}. 2001.

\bibitem{hadash2018estimate}
Guy Hadash, Einat Kermany, Boaz Carmeli, Ofer Lavi, George Kour, and Alon
  Jacovi.
\newblock Estimate and replace: A novel approach to integrating deep neural
  networks with existing applications.
\newblock {\em arXiv preprint arXiv:1804.09028}, 2018.

\end{thebibliography}


\end{document}
